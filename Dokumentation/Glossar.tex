% !TeX root = Dokumentation.tex

\newglossaryentry{git}{
	name={Git}, 
	description={
		Quelle: \href{https://git-scm.com/book/en/v2F}{https://git-scm.com/book/en/v2}\newline
		Git ist ein Open-Source Versionskontrollsystem, dass eine übersichtliche Historie der Änderungen hält und es Entwicklern ermöglicht unabhängig voneinander zu arbeiten. Dafür benutzt Git \glspl{Branch} und \glspl{Commit}}
}
	
\newglossaryentry{Commit}{
	name={Commit}, 
	description={
		Quelle: \href{https://de.wikipedia.org/wiki/Commit}{https://de.wikipedia.org/wiki/Commit}\newline
		Ein Commit ist eine Momentaufnahme der Software und enthält die Änderungen, die seit dem letzten Commit hinzugefügt wurden},
	plural=Commits
}

\newglossaryentry{Branch}{
	name={Branch}, 
	description={
		Quelle: \href{https://git-scm.com/book/en/v2/Git-Branching-Branches-in-a-Nutshell}{https://git-scm.com/book/en/v2/Git-Branching-Branches-in-a-Nutshell}\newline
		Ein Branch enthält eine Menge von Commits und kann als Zeitstrahl gesehen werden. Es können mehrere Branches zeitgleich existieren. Dadurch können Entwickler vollkommen unabhängig voneinander, zeitgleich an der gleichen Software arbeiten},
	plural=Branches
}

\newglossaryentry{Docker}{
	name={Docker}, 
	description={
		Quelle: \href{https://de.wikipedia.org/wiki/Docker\_(Software)}{https://de.wikipedia.org/wiki/Docker\_(Software)}\newline
		Docker ist eine Open Source Anwendung, die betriebssystemunabhängig Funktionalitäten für die Ausführung, Erstellung und Verbreitung von isolierten Anwendungen zur Verfügung}
}

\newglossaryentry{Image}{
	name={Image}, 
	description={
		Quelle:\newline \href{https://jfrog.com/de/devops-tools/article/understanding-and-building-docker-images/}{https://jfrog.com/de/devops-tools/article/understanding-and-building-docker-images/}\newline
		Ein Image ist eine Vorlage um Container zu erstellen. Es enthält alle erforderlichen Dateien und Konfigurationen für eine Anwendung},
	plural=Images
}

\newglossaryentry{Container}{
	name={Container}, 
	description={
		Quelle: \href{https://jfrog.com/devops-tools/article/what-are-containers/}{https://jfrog.com/devops-tools/article/what-are-containers/}\newline
		Ein Container ist eine abgekapselte Anwendung, die von Docker ausgeführt werden kann. ein Image in Ausführung. Das bedeutet, dass ein Container eine abgekapselte, laufende Anwendung ist. Ein Container enthält zudem nur das mindeste, was für die Ausführung der Anwendung nötig ist. Alle Container laufen auf dem Kernel des Host-Systems},
	plural=Containern
}

\newglossaryentry{Docker-Hub}{
	name={Docker-Hub}, 
	description={
		Quelle: \href{https://docs.docker.com/docker-hub/}{https://docs.docker.com/docker-hub/}\newline
		Docker Hub ist eine Plattform, auf der Images hoch- und heruntergeladen werden können}
}

\newglossaryentry{Releaseprozess}{
	name={Releaseprozess}, 
	description={Ein Releaseprozess ist ein Prozess, bei dem etwas in das Artifactory hochgeladen wird}
}

\newglossaryentry{Maven-Artefakt}{
	name={Maven-Artefakt}, 
	description={
		Quelle: \href{https://maven.apache.org/repositories/artifacts.html}{https://maven.apache.org/repositories/artifacts.html}\newline
		Ein Maven-Artefakt ist eine Datei, die von Maven adressiert werden kann},
	plural=Maven-Artefakte,
}

\newglossaryentry{Event}{
	name={Event}, 
	description={
		Quelle: \href{https://de.wikipedia.org/wiki/Ereignis\_(Programmierung)}{https://de.wikipedia.org/wiki/Ereignis\_(Programmierung)}\newline
		Ein Event ist ein Ereignis, das den Programmfluss steuert. In der Regel gibt es für ein Event auch Event-Listener},
	plural=Events
}

\newglossaryentry{Event-Listener}{
	name={Event-Listener}, 
	description={
		Quelle: \href{https://de.wikipedia.org/wiki/Ereignis\_(Programmierung)}{https://de.wikipedia.org/wiki/Ereignis\_(Programmierung)}\newline
		Ein Event-Listener ist eine Komponente, die auf ausgewählte Events reagiert}
}

\newglossaryentry{JFrog-Artifactory}{
	name={JFrog-Artifactory}, 
	description={
		Quelle: \href{https://jfrog.com/de/artifactory/}{https://jfrog.com/de/artifactory/}\newline
		Das JFrog-Artifactory ist ein Service, der viele verschiedene Dateien, wie Maven-Artefakte und Images verwaltet und bereitstellt},
	alias={Artifactory}
}

\newglossaryentry{Persona}{
	name={Persona}, 
	description={
		Quelle: \href{https://de.wikipedia.org/wiki/Persona\_(Mensch-Computer-Interaktion)}{https://de.wikipedia.org/wiki/Persona\_(Mensch-Computer-Interaktion)}\newline
		Eine Persona ist eine ausgedachte Person, die repräsentativ für eine Benutzergruppe steht},
	plural=Personas
}

\newglossaryentry{Anwendungsfall}{
	name={Anwendungsfall}, 
	description={
		Quelle: \href{https://de.wikipedia.org/wiki/Anwendungsfall}{https://de.wikipedia.org/wiki/Anwendungsfall}\newline
		Ein Anwendungsfall ist ein Ziel das ein Benutzer erreichen möchte, wenn er mit dem System interagiert},
	plural=Events
}

\newglossaryentry{Szenario}{
	name={Szenario}, 
	description={
		Ein Szenario ist ein konkrter Handlungsstrang, der von dem Benutzer des Systems ausgeführt werden kann um ein Ziel zu erreichen},
	plural=Szenarien
}

\newglossaryentry{Funktionale Anforderung}{
	name={funktionale Anforderung}, 
	description={
		Quelle:\newline\href{https://de.wikipedia.org/wiki/Anforderung\_\%28Informatik\%29}{https://de.wikipedia.org/wiki/Anforderung\_\%28Informatik\%29}\newline
		Eine funktionale Anforderung legt das Verhalten des Systems fest}
}

\newglossaryentry{nicht-funktionale Anforderung}{
	name={Nicht Funktionale Anforderung}, 
	description={
		Quelle:\newline\href{https://de.wikipedia.org/wiki/Anforderung\_\%28Informatik\%29}{https://de.wikipedia.org/wiki/Anforderung\_\%28Informatik\%29}\newline
		Eine nicht-funktionale Anforderung beschreibt Eigenschaften des Systems},
	plural=nicht-funktionale Anforderungen
}

\newglossaryentry{Java}{
	name={Java}, 
	description={
		Quelle: \href{https://de.wikipedia.org/wiki/Java\_(Programmiersprache)}{https://de.wikipedia.org/wiki/Java\_(Programmiersprache)}\newline
		Java ist eine objektorientierte Programmiersprache}
}

\newacronym{re}{RE}{Requirements Engineering}

\newacronym{gui}{GUI}{Graphical User Interface}

\newacronym{poc}{PoC}{Proof of Concept}

\glsaddall